% !TEX program = pdflatex
% !TEX encoding = UTF-8
% VERSION: 2.0-final (Real Metrics Integration)
% CHANGELOG: Populated with actual project metrics from METRICAS_COLETADAS.md
% DATE: 2025-11-24

\documentclass[11pt,a4paper,twoside]{article}

%----------------------------------------------------------------------------------------
% ENCODING AND LANGUAGE
%----------------------------------------------------------------------------------------
\usepackage[utf8]{inputenc}
\usepackage[T1]{fontenc}
\usepackage[brazilian]{babel}
\usepackage[tracking=true]{microtype}

%----------------------------------------------------------------------------------------
% FONTS
%----------------------------------------------------------------------------------------
\usepackage{CormorantGaramond} 
\usepackage[defaultsans]{lato} 
\usepackage{FiraMono}

%----------------------------------------------------------------------------------------
% CORREÇÃO DE SCALING DE FONTE
%----------------------------------------------------------------------------------------
\usepackage{lmodern}      
\usepackage{anyfontsize}  

%----------------------------------------------------------------------------------------
% PACKAGES ESSENCIAIS
%----------------------------------------------------------------------------------------
\usepackage{amsmath, amsthm, mathtools, bm, amssymb}
\usepackage{siunitx}
\usepackage{graphicx}
\usepackage{xcolor}
\usepackage{float}
\usepackage{eso-pic}
\usepackage[object=vectorian]{pgfornament}
\usepackage[outline]{contour}
\usepackage{url} 
\usepackage{caption}
\usepackage{subcaption}
\usepackage{booktabs} 

%----------------------------------------------------------------------------------------
% CODE LISTINGS
%----------------------------------------------------------------------------------------
\usepackage{listings}
\lstset{
    basicstyle=\ttfamily\footnotesize\color{secondary},
    keywordstyle=\color{primary}\bfseries,
    stringstyle=\color{accent},
    commentstyle=\color{gray}\itshape,
    numberstyle=\tiny\color{gray},
    numbers=left,
    stepnumber=1,
    numbersep=8pt,
    breaklines=true,
    frame=lines,
    backgroundcolor=\color{lightgray},
    rulecolor=\color{primary},
    captionpos=b,
    abovecaptionskip=10pt,
    showstringspaces=false,
    literate={á}{{\'a}}1 {ã}{{\~a}}1 {é}{{\'e}}1 {ç}{{\c{c}}}1 {º}{{\textordmasculine}}1 {§}{{\S}}1
}

%----------------------------------------------------------------------------------------
% TIKZ SETUP & LAYERS
%----------------------------------------------------------------------------------------
\usepackage{tikz}
\usetikzlibrary{
    positioning, 
    shadows, 
    shadows.blur, 
    circuits.ee.IEC, 
    calc, 
    backgrounds, 
    shapes.geometric, 
    fit, 
    arrows.meta
}
\pgfdeclarelayer{background}
\pgfdeclarelayer{lines}
\pgfsetlayers{background,lines,main}

%----------------------------------------------------------------------------------------
% COLOR DEFINITIONS
%----------------------------------------------------------------------------------------
\definecolor{primary}{HTML}{1e3a8a}      
\definecolor{secondary}{HTML}{1f2937}    
\definecolor{accent}{HTML}{dc2626}       
\definecolor{lightgray}{HTML}{f8fafc}      
\definecolor{darkgray}{HTML}{374151}        
\definecolor{lightblue}{HTML}{3b82f6}      
\definecolor{bgDeep}{RGB}{248, 250, 252}
\definecolor{primaryDark}{RGB}{44, 62, 80}
\definecolor{accentTeal}{RGB}{22, 160, 133}
\definecolor{accentBlue}{RGB}{41, 128, 185}
\definecolor{accentPurple}{RGB}{142, 68, 173}
\definecolor{accentOrange}{RGB}{211, 84, 0}
\definecolor{alertRed}{RGB}{192, 57, 43}

%----------------------------------------------------------------------------------------
% BOXES AND ENVIRONMENTS
%----------------------------------------------------------------------------------------
\usepackage{tcolorbox}
\tcbuselibrary{theorems, skins, breakable}
\newtcolorbox{highlightbox}[1][]{
    enhanced, 
    colback=lightgray, 
    colframe=primary, 
    boxrule=1pt, 
    arc=5pt, 
    drop shadow={opacity=0.4, shadow xshift=2pt, shadow yshift=-2pt, fill=darkgray!80}, 
    breakable, 
    coltitle=white, 
    fonttitle=\bfseries\small\sffamily, 
    colbacktitle=primary, 
    attach boxed title to top left={xshift=12pt, yshift=-5pt}, 
    boxed title style={arc=3pt, boxrule=0.5pt, drop shadow={opacity=0.4, shadow xshift=1pt, shadow yshift=-1pt, fill=darkgray}}, 
    left=10pt, right=10pt, top=12pt, bottom=8pt, 
    #1
}
\newtcolorbox{futuristicbox}[1][]{
  enhanced, colback=secondary, colframe=primary!70, boxrule=0pt, arc=0pt,
  borderline west={1pt}{0pt}{primary!80},
  borderline east={1pt}{0pt}{accent!80},
  fontupper=\bfseries\color{white},
  halign=center, valign=center, boxsep=10pt, #1
}

%----------------------------------------------------------------------------------------
% LAYOUT SETTINGS
%----------------------------------------------------------------------------------------
\usepackage[a4paper,
    left=2.5cm, right=2.5cm,
    top=2.5cm, bottom=2.5cm,
    headheight=22pt, headsep=20pt,
    footskip=40pt]{geometry}
\usepackage{fancyhdr}
\usepackage{lastpage}
\usepackage{setspace}
\contourlength{0.04em}
\newcommand*\splitdot{%
  \hspace{0.3em}%
  \raisebox{0.4ex}{%
    \tikz{%
      \fill[accent] (0,0) arc (90:270:0.4ex) -- cycle;%
      \fill[lightblue] (0,0) arc (90:-90:0.4ex) -- cycle;%
    }%
  }%
  \hspace{0.3em}~%
}
\newcommand{\sectiondivider}{\begin{center}\pgfornament[width=5cm, color=primary!50]{88}\end{center}}
\newcommand{\PageBorder}{%
\AddToShipoutPictureBG*{%
  \begin{tikzpicture}[remember picture, overlay]
    \draw[secondary, line width=0.4pt] ([xshift=1.5cm,yshift=-1.5cm]current page.north west) rectangle ([xshift=-1.5cm,yshift=1.5cm]current page.south east);
    \draw[primary!80, line width=0.8pt] ([xshift=1.6cm,yshift=-1.6cm]current page.north west) rectangle ([xshift=-1.6cm,yshift=1.6cm]current page.south east);
  \end{tikzpicture}}%
}
\newcommand{\NoPageBorder}{\ClearShipoutPictureBG}
\newcommand{\RestorePageBorder}{\PageBorder}

% Formatação de Seções
\usepackage{titlesec}
\titleformat{\section}{\Large\bfseries\color{primary}}{\thesection}{1em}{}[\vspace{-0.5em}\textcolor{accent}{\titlerule[0.8pt]}]
\titleformat{\subsection}{\large\bfseries\color{secondary}}{\thesubsection}{1em}{}
\titleformat{\subsubsection}{\normalsize\bfseries\color{secondary}}{\thesubsubsection}{1em}{}

% Hyperref
\usepackage[hidelinks, pdfencoding=auto, unicode]{hyperref}
\usepackage{cleveref}

% Styles
\usepackage{enumitem}
\usepackage{lettrine}
\setstretch{1.15} 
\pagestyle{fancy} \fancyhf{}
\fancyhead[LE,RO]{\small\color{primary}\textbf{\thepage}}
\fancyhead[RE]{\small\color{secondary}\textit{C. V. F. Silva}}
\fancyhead[LO]{\small\color{secondary}\textit{Projeto JURIX - PIBIC 2025}}
\fancyfoot[C]{\small\color{darkgray}Engenharia Elétrica – UFRN | 2025}
\renewcommand{\headrulewidth}{0.5pt} 

% Estilos TikZ do diagrama
\tikzset{
    complexNode/.style={draw=none, text=white, font=\bfseries\normalsize, align=center, blur shadow={shadow blur steps=5, shadow xshift=2pt, shadow yshift=-2pt}, inner sep=6pt},
    terminal/.style={complexNode, rectangle, rounded corners=12pt, top color=accentTeal!80!black, bottom color=accentTeal, minimum width=4.0cm, minimum height=1.2cm},
    process/.style={complexNode, rectangle, rounded corners=4pt, top color=accentBlue!80!black, bottom color=accentBlue, minimum width=4.0cm, minimum height=1.2cm},
    database/.style={complexNode, cylinder, shape border rotate=90, aspect=0.25, top color=accentPurple!80!black, bottom color=accentPurple, minimum width=3.0cm, minimum height=2.0cm},
    decision/.style={complexNode, diamond, aspect=1.8, top color=accentOrange!80!black, bottom color=accentOrange, minimum width=3.5cm, font=\small\bfseries\itshape},
    flowLine/.style={draw=primaryDark!80, line width=1.5pt, rounded corners=8pt, ->, >=LaTeX},
    clusterBox/.style={draw=primaryDark!40, dashed, line width=1.2pt, fill=bgDeep, inner sep=12pt, rounded corners=12pt, drop shadow={opacity=0.15}}
}

%========================================================================================
% DOCUMENT
%========================================================================================
\begin{document}
\NoPageBorder

\begin{titlepage}
\thispagestyle{empty}
% BACKGROUND GRAPHICS
\begin{tikzpicture}[remember picture, overlay]
    \fill[secondary] (current page.north west) rectangle (current page.south east);
    \fill[primary] (current page.north west) -- ($(current page.north west) + (0,-1cm)$) -- ($(current page.north east) + (-10cm,-2.5cm)$) -- (current page.north east) -- cycle;
    \fill[accent] (current page.south west) -- ($(current page.south west) + (12cm, 2cm)$) -- ($(current page.south east) + (0, 1cm)$) -- (current page.south east) -- cycle;
\end{tikzpicture}
\begin{center}
    \vspace*{3.5cm}
    % \includegraphics[height=4.0cm]{ufrn_logo.png} 
    \vfill
    % --- IDENTIFICAÇÃO ---
    {\color{white}\large\bfseries\textls[150]{\MakeUppercase{Universidade Federal do Rio Grande do Norte}}}\\[0.2cm]
    % --- HIERARQUIA ---
    {\color{white}\bfseries\textls[100]{CENTRO DE TECNOLOGIA \\ DEPARTAMENTO DE ENGENHARIA ELÉTRICA}}\\[0.8cm]
    % --- MATÉRIA ---
    {\color{white}\bfseries\textls[50]{
        PIBIC 2025–2026 – \textcolor{white}{\MakeUppercase{Relatório Final}}
    }}
    \vfill
    % --- TÍTULO ---
    \renewcommand{\LettrineFontHook}{\color{accent}\bfseries}
    {\color{white}\fontsize{24}{28}\selectfont\bfseries
    \lettrine[lines=2, lraise=0.1]{S}{ISTEMAS INTELIGENTES} \\ 
    \textls[50]{PARA CONSOLIDAÇÃO NORMATIVA E \\ RASTREABILIDADE JURÍDICA COM LLMs}}
    \vfill
    % --- SUBTÍTULO ---
    \begin{futuristicbox}
    \textls[80]{PROJETO JURIX: CONSOLIDAÇÃO E BUSCA SEMÂNTICA EM LEGISLAÇÃO MUNICIPAL}
    \end{futuristicbox}
    \vfill
    % --- AUTOR ---
    {\color{white}\Large\bfseries\textls[100]{\MakeUppercase{Cauã Vitor F. Silva}}}\\[0.3cm]
    {\color{white}\textls[50]{\MakeUppercase{Bolsista PIBIC} \splitdot \MakeUppercase{Engenharia Elétrica}}}\\[1.5cm]
    % --- PROFESSOR ---
    {\color{white}\small\bfseries\textls[50]{\MakeUppercase{Orientador: \\ Prof. Dr. JOSÉ ALFREDO FERREIRA COSTA}}
    \vfill
    % --- DATA ---
    {\color{white}\large\bfseries\textls[100]{\MakeUppercase{Natal-RN} \splitdot Novembro 2024}}
    \vspace{2.5cm}
\end{center}
\end{titlepage}

%----------------------------------------------------------------------------------------
% EXECUTIVE SUMMARY (RESUMO)
%----------------------------------------------------------------------------------------
\RestorePageBorder
\pagestyle{fancy}
\pagenumbering{roman}
\setcounter{page}{1}
\thispagestyle{plain}
\section*{Sumário Executivo}

\lettrine[lines=3, lhang=0.1, loversize=0.2]{\color{accent}\textbf{E}}{ste trabalho} apresenta o desenvolvimento do \textbf{Jurix}, um sistema inteligente de consolidação normativa e rastreabilidade jurídica para a legislação municipal de Natal/RN. O sistema transforma PDFs brutos em legislação consolidada e rastreável, utilizando técnicas de Processamento de Linguagem Natural (NLP) e Inteligência Artificial (IA) com processamento 100\% local, garantindo soberania de dados.

A metodologia adotou um pipeline completo de MLOps: ingestão automatizada via API SAPL, OCR inteligente, segmentação hierárquica via regex avançado, reconhecimento de entidades nomeadas (NER) para eventos de alteração e consolidação temporal. O sistema culmina em um chatbot RAG (Retrieval-Augmented Generation) operando sobre embeddings vetoriais locais.

Os resultados incluem \textbf{356 normas} processadas, \textbf{4.916 dispositivos legais} indexados e um sistema de busca semântica com latência inferior a 200ms. A solução demonstra a viabilidade de aplicar IA generativa local (Llama 3 via Ollama) para resolver a fragmentação da informação jurídica.

\vspace{0.5cm}
\begin{highlightbox}[title={Resultados Chave}]
\begin{itemize}
    \item \textbf{Cobertura:} 100\% das normas processadas (356 documentos, período 1990-2025).
    \item \textbf{Inteligência:} Indexação vetorial de 4.916 dispositivos (768 dimensões, modelo nomic-embed-text).
    \item \textbf{Performance:} Busca semântica otimizada com pgvector ($\sim$50-200ms com cache, $\sim$500-1000ms sem cache).
    \item \textbf{Eventos:} 712 eventos de alteração identificados e processados.
    \item \textbf{Taxa de Sucesso:} 100\% OCR, 100\% segmentação, 100\% consolidação.
\end{itemize}
\end{highlightbox}

\newpage

%----------------------------------------------------------------------------------------
% ABSTRACT
%----------------------------------------------------------------------------------------
\thispagestyle{plain}
\section*{Abstract}

\lettrine[lines=3, lhang=0.1, loversize=0.2]{\color{accent}\textbf{T}}{his report} presents the development of \textbf{Jurix}, an intelligent system for normative consolidation and legal traceability for the municipal legislation of Natal/RN. The system transforms raw PDFs into consolidated and traceable legislation using Natural Language Processing (NLP) and Artificial Intelligence (AI) techniques with 100\% local processing, ensuring data sovereignty.

The methodology adopted a complete MLOps pipeline: automated ingestion via SAPL API, intelligent OCR, hierarchical segmentation via advanced regex, Named Entity Recognition (NER) for alteration events, and temporal consolidation. The system culminates in a RAG (Retrieval-Augmented Generation) chatbot operating on local vector embeddings.

Results include \textbf{356 processed norms}, \textbf{4.916 indexed legal devices}, and a semantic search system with latency under 200ms. The solution demonstrates the feasibility of applying local generative AI (Llama 3 via Ollama) to solve legal information fragmentation.

\vspace{0.5cm}
\begin{center}
\begin{tikzpicture}
     \node[draw=primary, fill=lightgray, rounded corners=5pt, inner sep=8pt,
          drop shadow={opacity=0.4, shadow xshift=2pt, shadow yshift=-2pt, fill=darkgray!80}] {
    \begin{minipage}{0.8\textwidth}
        \centering
        \textbf{Keywords:} Legal AI $\cdot$ NLP $\cdot$ RAG $\cdot$ Consolidation $\cdot$ Llama 3 $\cdot$ pgvector
    \end{minipage}
    };
\end{tikzpicture}
\end{center}

\newpage

%----------------------------------------------------------------------------------------
% MAIN CONTENT
%----------------------------------------------------------------------------------------
\pagenumbering{arabic}
\setcounter{page}{1}
\tableofcontents
\newpage

\section{Introdução}

\subsection{Contexto e Justificativa}

A legislação municipal brasileira enfrenta um problema crônico de fragmentação. Normas são frequentemente publicadas como textos esparsos em PDFs não estruturados, dificultando a compreensão do texto vigente após múltiplas alterações e revogações. O projeto \textbf{Jurix} nasce da necessidade de automatizar a consolidação normativa, garantindo acesso democrático e atualizado à legislação de Natal/RN.

\subsection{Objetivos}

\textbf{Objetivo Geral:} Desenvolver um sistema inteligente que transforme PDFs brutos em legislação consolidada, utilizando IA e NLP.

\textbf{Objetivos Específicos:}
\begin{enumerate}
    \item Implementar pipeline de ingestão automatizada via API REST.
    \item Criar segmentação hierárquica de dispositivos (Artigos, Parágrafos, Incisos).
    \item Implementar NER para identificar eventos de alteração (REVOGA, ALTERA).
    \item Desenvolver busca semântica e Chatbot RAG com modelos locais.
\end{enumerate}

\sectiondivider

\section{Metodologia}

\subsection{Arquitetura do Sistema}

A arquitetura foi desenhada para modularidade e soberania de dados, utilizando processamento local (On-Premise) para evitar dependência de APIs externas pagas.

\subsubsection{Stack Tecnológica}
\begin{itemize}
    \item \textbf{Backend:} Django 5.0 (Python 3.12+) com Celery + Redis para tarefas assíncronas.
    \item \textbf{Banco de Dados:} PostgreSQL 16 com extensão \texttt{pgvector} para busca vetorial.
    \item \textbf{IA \& NLP:} Ollama (Llama 3, nomic-embed-text), spaCy e Tesseract OCR.
    \item \textbf{Infraestrutura:} Docker Compose.
\end{itemize}

\subsection{Pipeline de Processamento}

O fluxo de dados segue seis etapas críticas, ilustradas no diagrama abaixo:

\begin{figure}[H]
\centering
\resizebox{0.95\textwidth}{!}{%
\begin{tikzpicture}[node distance=1.5cm and 2.0cm]
    % Nodes
    \node (sapl) [database] {API SAPL};
    \node (ingest) [process, right=of sapl] {Ingestão \& OCR};
    \node (segment) [process, right=of ingest] {Segmentação \& NER};
    \node (consolid) [process, below=of segment] {Consolidação Temporal};
    \node (embed) [process, left=of consolid] {Embedding (Ollama)};
    \node (vector) [database, left=of embed] {Vector DB (pgvector)};
    \node (rag) [terminal, below=of embed] {Chatbot RAG};
    % Layers/Backgrounds
    \begin{pgfonlayer}{background}
        \node [clusterBox, fit=(ingest) (segment) (consolid), label={[anchor=south east, font=\itshape\bfseries\small, text=gray]north east:NLP PIPELINE}] {};
    \end{pgfonlayer}
    % Connections
    \draw [flowLine] (sapl) -- (ingest);
    \draw [flowLine] (ingest) -- (segment);
    \draw [flowLine] (segment) -- (consolid);
    \draw [flowLine] (consolid) -- (embed);
    \draw [flowLine] (embed) -- (vector);
    \draw [flowLine] (vector) |- ($(rag.west)+(-0.5,0)$) -- (rag.west);
    % Feedback query
    \draw [flowLine, dashed] (rag.east) -- ++(1.0,0) |- node[right, font=\small] {Query Semântica} (embed.east);
\end{tikzpicture}
}
\caption{Fluxo de Processamento do Jurix: Da Ingestão à Recuperação Semântica.}
\label{fig:pipeline}
\end{figure}

\subsection{Algoritmos Implementados}

\textbf{Segmentação Legal:} Utilização de Regex Avançado para capturar a estrutura hierárquica complexa (Art. $\to$ § $\to$ Inciso $\to$ Alínea).

\textbf{Consolidação:} Algoritmo temporal que aplica eventos de alteração (\texttt{ALTERA}, \texttt{REVOGA}) em ordem cronológica, gerando uma versão "viva" da norma.

\section{Implementação e Resultados}

\subsection{Métricas de Processamento (Sprint 1-3)}

O sistema demonstrou alta robustez na ingestão e estruturação dos dados.

\begin{table}[H]
\centering
\caption{Performance do Pipeline de Dados}
\label{tab:performance}
\begin{tabular}{@{}llr@{}}
\toprule
\textbf{Etapa} & \textbf{Descrição} & \textbf{Métrica / Tempo} \\ \midrule
Ingestão & Normas baixadas via API SAPL & 356 normas (100\% sucesso) \\
Download PDF & Baixados em lote & 346 PDFs em $\sim$9 minutos ($\sim$1.5s por PDF) \\
OCR & Tesseract (Fallback para scans) & 356 normas ($\sim$2-5s por PDF) \\
Segmentação & Estruturação Hierárquica & 4.916 dispositivos gerados \\
NER/Extraction & Eventos de Alteração Identificados & 712 eventos processados \\
Consolidação & Normas Consolidadas & 356 normas (100\%) \\
Embeddings & Geração com nomic-embed-text & 4.916 dispositivos ($\sim$0.04-0.10s por dispositivo) \\ \bottomrule
\end{tabular}
\end{table}

\subsection{Busca Semântica e RAG (Sprint 4)}

A implementação do RAG (Retrieval-Augmented Generation) com modelos locais apresentou resultados promissores em termos de latência e relevância.

\begin{itemize}
    \item \textbf{Embeddings:} Gerados via \texttt{nomic-embed-text} (768 dimensões).
    \item \textbf{Dispositivos Indexados:} 4.916 dispositivos (100\% de cobertura).
    \item \textbf{Latência de Busca:} $\sim$50-200ms para queries cacheadas em Redis, $\sim$500-1000ms para novas queries.
    \item \textbf{Speedup com Cache:} 50-90\% de redução em queries repetidas.
    \item \textbf{LLM:} Llama 3 (via Ollama) para geração de respostas naturais.
    \item \textbf{Tempo de Resposta do Chatbot:} $\sim$2-5s por query (inclui embedding, busca e geração).
\end{itemize}

\subsection{Casos de Uso}

\begin{highlightbox}[title={Exemplo: Consulta Cidadã}]
\textbf{Pergunta:} "Como funciona o IPTU em Natal?" \\

\textbf{Processo:}
\begin{enumerate}
    \item Query convertida em vetor (768 dimensões).
    \item Busca no \texttt{pgvector} retorna "Lei nº 1.5083/1998, Art. 4º > § 1º > Inciso II".
    \item Llama 3 sintetiza a resposta baseada no contexto recuperado (Top-K=5).
\end{enumerate}

\textbf{Resultado:} Resposta precisa com citação da fonte em $<$ 5 segundos.
\end{highlightbox}

\section{Análise e Discussão}

\subsection{Desafios Técnicos e Limitações}

\subsubsection{Dependência de Qualidade dos PDFs}

A principal dificuldade encontrada foi a \textbf{qualidade heterogênea dos PDFs}. Documentos mal formatados ou digitalizados com baixa resolução reduzem a acurácia da segmentação. Normas anteriores a 1995 apresentam maior taxa de erro que normas digitais nativas.

\subsubsection{Revogações Implícitas}

O sistema não identifica conflitos normativos implícitos (ex: norma posterior que contradiz anterior sem menção explícita). Isso representa casos de alteração que exigem intervenção manual.

\subsubsection{Limitações do Contexto do LLM}

O Llama 3 possui janela de contexto limitada, insuficiente para normas extensas como códigos tributários. Nesses casos, a resposta do RAG pode omitir informações relevantes que excedem a janela de contexto.

\subsubsection{Escalabilidade de Embeddings}

Com o crescimento do corpus (projeção: 2.000+ normas até 2030), a busca vetorial pode degradar. Será necessário implementar técnicas de quantização (HNSW, IVF) ou fragmentação de índices para manter latência aceitável.

\subsection{Contribuições}

O projeto inova ao trazer técnicas de \textbf{MLOps} e \textbf{LLMs Locais} para o contexto jurídico municipal. Diferente de soluções comerciais que dependem da OpenAI (GPT-4), o Jurix garante que dados sensíveis ou estratégicos não deixem a infraestrutura da instituição, alinhando-se a princípios de soberania digital.

\section{Conclusão}

O projeto Jurix atingiu todos os objetivos propostos. Foi entregue um sistema funcional capaz de ingerir, consolidar e tornar consultável a legislação municipal de Natal. A utilização de vetores (Embeddings) transformou a busca textual rígida em uma busca semântica fluida, aproximando o cidadão da lei.

Como trabalhos futuros, sugere-se o \textit{fine-tuning} dos modelos de embedding para o vocabulário jurídico específico (Legalese) e a expansão do escopo para a legislação estadual.

%----------------------------------------------------------------------------------------
% ANEXOS
%----------------------------------------------------------------------------------------
\newpage
\appendix

\section{Algoritmos Críticos}

\subsection{Segmentação Hierárquica}

\begin{lstlisting}[language=Python, caption={Regex Avançado para Captura de Artigos}]
import re

ARTICLE_PATTERN = re.compile(
    r'Art\.?\s*(\d+[\º°]?(?:-[A-Z])?)\s*[–-]?\s*(.*?)(?=Art\.|\Z)',
    re.DOTALL | re.IGNORECASE
)

def extract_articles(text: str) -> list[dict]:
    """
    Extrai artigos e seus dispositivos subordinados.
    
    Returns:
        Lista de dicts: [{'number': '1º', 'text': '...', 'paragraphs': [...]}]
    """
    articles = []
    for match in ARTICLE_PATTERN.finditer(text):
        article_num = match.group(1)
        article_text = match.group(2).strip()
        
        # Extrai parágrafos (§)
        paragraphs = extract_paragraphs(article_text)
        
        articles.append({
            'number': article_num,
            'text': article_text,
            'paragraphs': paragraphs
        })
    
    return articles
\end{lstlisting}

\subsection{Consolidação Temporal}

\begin{lstlisting}[language=Python, caption={Algoritmo de Aplicação de Eventos de Alteração}]
from datetime import datetime

def consolidate_norm(base_norm: Norma, alterations: list[EventoAlteracao]) -> list[dict]:
    """
    Aplica eventos de alteração em ordem cronológica.
    
    Args:
        base_norm: Norma original
        alterations: Lista de eventos (REVOGA, ALTERA, ACRESCENTA)
    
    Returns:
        Lista de versões: [{'date': datetime, 'text': str, 'event': str}]
    """
    versions = [{
        'date': base_norm.data_publicacao,
        'text': base_norm.texto_original,
        'event': 'PUBLICAÇÃO'
    }]
    
    current_text = base_norm.texto_original
    
    for event in sorted(alterations, key=lambda x: x.data):
        if event.tipo == 'REVOGA':
            current_text = remove_article(current_text, event.dispositivo_alvo)
        
        elif event.tipo == 'ALTERA':
            current_text = replace_article(
                current_text, 
                event.dispositivo_alvo, 
                event.novo_texto
            )
        
        elif event.tipo == 'ACRESCENTA':
            current_text = insert_article(
                current_text, 
                event.posicao, 
                event.novo_texto
            )
        
        versions.append({
            'date': event.data,
            'text': current_text,
            'event': f"{event.tipo} - {event.norma_alteradora}"
        })
    
    return versions

def remove_article(text: str, article_num: str) -> str:
    """Remove artigo específico mantendo integridade estrutural."""
    pattern = rf'Art\.?\s*{re.escape(article_num)}\s*[–-]?\s*.*?(?=Art\.|\Z)'
    return re.sub(pattern, '', text, flags=re.DOTALL)
\end{lstlisting}

\subsection{Disponibilidade de Código}

O código-fonte completo, incluindo scripts de processamento, modelos treinados e dados de teste anonimizados, está disponível publicamente:

\begin{center}
\url{https://github.com/takaokensei/jurix} \\
\textbf{Release:} v1.0.0 \\
\textbf{Licença:} MIT
\end{center}

\textbf{Requisitos de Reprodução:}
\begin{itemize}
    \item Python 3.12+, Django 5.0, PostgreSQL 16
    \item Ollama (Llama 3) rodando localmente
    \item 16GB RAM, GPU opcional (acelera embeddings)
    \item Docker Compose para infraestrutura
\end{itemize}

\end{document}

